	\documentclass[10pt,conference]{IEEEtran}

\usepackage{cite}
\ifCLASSINFOpdf
   \usepackage[pdftex]{graphicx}
   \graphicspath{{figs/}}
   \DeclareGraphicsExtensions{.pdf,.jpeg,.png}
\else
   \usepackage[dvips]{graphicx}
   \graphicspath{{../figs/}}
   \DeclareGraphicsExtensions{.eps}
\fi

\usepackage[cmex10]{amsmath}
\interdisplaylinepenalty=2500
\usepackage{amsthm}
\newtheorem{definition}{Definition}
\usepackage{algorithmic}
\usepackage{array}
\usepackage{subcaption}
\usepackage{url}
\usepackage[T1]{fontenc}
\usepackage[utf8]{inputenc}
\usepackage[brazilian]{babel}

% corrija hifenação aqui
\hyphenation{op-tical net-works semi-conduc-tor}

\begin{document}
\title{Modelo de Relatório - Robótica Computacional}

\newif\iffinal
\finalfalse
\finaltrue
\newcommand{\jemsid}{99999}

\iffinal
\author{\IEEEauthorblockN{Aluno 1}
\IEEEauthorblockA{Engenharia da Computação \\
INSPER\\
Email: aluno1@insper.edu.br}
\and
\IEEEauthorblockN{Aluno 2}
\IEEEauthorblockA{Engenharia da Computação\\
INSPER\\
Email: aluno2@insper.edu.br}
}

\else
  \author{Sibgrapi paper ID: \jemsid \\ }
\fi


\maketitle

\begin{abstract}
Um resumo de 100-200 palavras do seu trabalho. Deve conter, brevemente, motivação, problema estudado, solução proposta/implementada e resultados obtidos.
\end{abstract}
\IEEEpeerreviewmaketitle

\section{Introdução}

Sua introdução deve responder às seguintes perguntas:

\begin{enumerate}
\item Qual a motivação do trabalho? (Por que o problema estudado é útil?)
\item Qual é o problema estudado?
\item Quais dificuldades existem?
\item Qual a solução proposta e como ela se relaciona com as dificuldades existentes?
\item Como a solução será avaliada? (Quais testes serão feitos para mostrar que o projeto funciona?)
\end{enumerate}

\section{O problema}

Mude o título para o nome do problema que vocês estão tratando e descreva ela aqui em mais detalhes. Se vocês estão se baseando em um método existente, é muito importante deixar isto claro tanto na introdução quanto aqui. 

Você pode adicionar aqui referências a trabalhos que tratam do mesmo problema, mas com abordagens diferentes. Por exemplo, você pode usar o comando ``cite'' para citar o software OpenCV~\cite{itseez2015opencv}.

\section{Método/Solução proposto(a)}

Descreva aqui com detalhes tudo o que foi realizado no projeto. Lembre-se que é mais importante descrever a ideia que foi usada para criar o código do que o código em si. A descrição de como o projeto está organizado fica no repositório do código. 

\section{Resultados / Avaliação dos resultados}

Descreva nesta seção quais são os resultados esperados de maneira mais detalhada e o que vocês farão para verificar se os resultados foram cumpridos. É importante pensar em como medir o sucesso do projeto (qual porcentagem de vezes o robô realizou a tarefa corretamente? se é projeto de localização, qual o erro entre a posição real e a estimada? etc ) e analisar com alguma profundidade os erros cometidos (se errou, qual parte do método proposto falhou?)

Exemplo:

Para testar a eficácia do método proposto na tarefa XYZ realizaremos os seguintes experimentos:

\begin{enumerate}
\item executar tarefa no contexto ABC;
\item testar robustez à situação A;
\item verificar se o método funciona na situação B.
\end{enumerate}

No contexto ABC o método proposto resolveu a tarefa corretamente em 4 de 5 execuções. Na execução sem sucesso a parte ABC do método não funcionou.

\section{Conclusão}
Relembre os pontos descritos na introdução e avalie se eles foram cumpridos. Relembre os pontos positivos e negativos do método usado no projeto.

\section*{Recursos para aprender \LaTeX}

Os dois sites abaixo são excelentes fontes para aprender \LaTeX e devem ser consultados antes de perguntar para os professores ;)

\begin{itemize}
\item https://en.wikibooks.org/wiki/LaTeX
\item https://tex.stackexchange.com
\end{itemize}

\bibliographystyle{IEEEtran}
\bibliography{example}
\end{document}


